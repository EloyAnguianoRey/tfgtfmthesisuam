Para referenciar elementos etiquetados en el documento se pueden utilizar cualquiera de los siguientes comandos:

\begin{description}
  \item [\textbackslash ref\{label\}] Indica la numeración del elemento indicado por la etiqueta.
  \item [\textbackslash pageref\{label\}] Es idéntico a \textbackslash ref pero indicando el número de la página.
  \item [\textbackslash cref, \textbackslash Cref, \textbackslash crefrange ] Comandos del paquete cleveref. En el caso de usarlos en su versión con * previenen de que sean hiperreferenciados.
\end{description}

Vamos a ver el uso de algunos de estos comandos en este párrafo. Como en todos los casos, se dispone de los fuentes para ver cómo se usan. Como ejemplo podemos referenciar a la sección \ref{SEC:CAPITULOS} o sin poner a mano lo de sección se puede referenciar como \cref{SEC:CAPITULOS} o, para finalizar, indicando que está en la página \pageref{SEC:CAPITULOS}.

El uso de los comandos del paquete cleveref pueden verse en su manual, sin embargo no es necesario que se busque dado que se dispone de un enlace a este manual en el \cref{CAP:PAQUETES}.
